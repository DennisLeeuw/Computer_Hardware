Fibre Channel draait over glasvezel, dit maakt het voor veel bedrijven onaantrekkelijk omdat de meeste bedrijven een koper infrastructuur met ethernet hebben liggen. Het voordeel van glas zijn hogere snelheden en de veel langere afstanden die overbrugd kunnen worden. De eerste standaard verscheen in 1993 en de laatste standaard, 256GFC stamt uit 2019. Fibre Channel is een netwerk protocol, OSI layers 1 t/m 4, waarover nog commando's gestuurd moeten worden om disken aan te sturen. Het meest gebruikte protocol daarvoor is SCSI.

\begin{center}
\begin{tabular}{ |c|c|c| }
\hline
Naam       & Lijnsnelheid & Throughput \\
           & (gigabaud)   & (MB/s) \\
\hline\hline
133 Mbit/s & 0,1328125    & 12,5 \\
\hline
266 Mbit/s & 0,265625     & 25 \\
\hline
1GFC       & 1,0625       & 100 \\
\hline
2GFC       & 2,125        & 200 \\
\hline
4GFC       & 4,25         & 400 \\
\hline
8GFC       & 8,5          & 800 \\
\hline
10GFC      & 10,51875     & 1.200 \\
\hline
16GFC      & 14,025       & 1.600 \\
\hline
32GFC      & 28,05        & 3.200 \\
\hline
64GFC      & 28,9         & 6.400 \\
\hline
128GC      & 28,05 x 4    & 12.800 \\
\hline
256GFC     & 28,08 x 4    & 25.600 \\
\hline
\end{tabular}
\end{center}
