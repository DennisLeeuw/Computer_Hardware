Er bestaat de mogelijkheid om meerdere CPU's op een moederbord te plaatsen. Dit heet Symmetric Multiprocessing (SMP). Het besturingssysteem moet SMP aware zijn, het ondersteunen, om gebruik te kunnen maken van de meerdere processoren die aanwezig zijn.

Om het moederbord minder complex te maken kunnen er ook meerdere dies in een behuizing gestopt worden. Je hebt dan meerdere cores in een CPU, ook dit is SMP en moet het OS weten hoe hiermee om te gaan. Het staat bekent als Chip Level Multiprocessing (CLM) en kan bestaan uit 2, 4, 8 of meerdere cores in een enkele CPU behuizing.

Verdere integratie vinden we bij meerdere cores op een die.

