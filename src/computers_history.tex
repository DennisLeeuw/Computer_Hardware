Het is moeilijk om een beginpunt in de automatisering aan te wijzen, maar het begin ligt van het computer tijdperk. Letterlijk vertaald betekent computer rekenaar, dus we zouden kunnen kijken naar de eerste rekenmachines die er ooit gemaakt zijn. De vraag die dan al snel opkomt is een telraam, of een stok waarop je kerven zet om te tellen een computer. Of moeten we kunnen spreken van invoer en uitvoer, zodat met de uitvinden van ponskaarten de feitelijke aanvang van de computer komt. Is een draai-orgel dan misschien een computer?

Misschien moeten we spreken van electronische computer, dus een computer die gebruik maakt van electronika. Dan kunnen we de Colossus aanwijzen als eerste computer. De Colossus stamt uit 1943 en werd in het Verenigd Koninkrijk ontwikkeld om encrypte berichten van de Duitsers te kraken. In Nederland werd de eerste computer in 1952 in gebruik genomen, dit was de ARRA I die alleen op de dag van introductie gewerkt heeft, de opvolger, ARRA II, uit 1954 was een stuk succesvoller.

De eerste electronische computers gebruikten buizen als schakelaars, met de uitvinding van de transistor in 1947 werd het mogelijk om veel kleinere computers te bouwen. De eerste electronische computers waren zo groot als 1 of meerdere klaslokalen. Met de transistor kon dat vele malen kleiner en de mogelijkheid om meerdere transistoren op een stukje silicium te stoppen (de IC: Integrated Circuit) zorgde ervoor dat het nog kleiner kon en dat de CPU (de centrale rekeneenheid van de computer) op een klein stukje silicium gestopt kon worden.

Al deze ontwikkelingen hebben in de jaren 1970-1980 geleidt tot de ontwikkeling van de thuis of home-computer. Een vaak nog zwaar apparaat dat op tafel stond en waarmee je eenvoudige applicaties als tekstverwerker en spreadsheet kon bedienen, maar waarop ook al snel spelletjes beschikbaar kwamen. De pioniers op dit gebied waren vooral Tandy, Acorn, Commodore, Apple en nog enkele leveranciers. Vele zijn van de markt verdwenen, zeker toen IBM bedacht dat deze machines ook gebruikt konden worden in de zakelijke markt. IBM licenceerde van het toen nog onbekende bedrijf Microsoft de software (PC-DOS) om hun IBM PC aan te sturen.
