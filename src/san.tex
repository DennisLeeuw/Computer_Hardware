Een Storage Attached network is zoals de naam al zegt een netwerk van storage systemen. Dit netwerk maakt het mogelijk de totale opslag te delen en te herverdelen met verschillende servers. Zo kan er effici\"enter met de ingekochte storage om gegaan worden. Storage kan ook van de ene host omgezet worden naar een andere host tijdens bijvoorbeeld een update van een server.

De meest gebruikte netwerktechnieken op een storage netwerk zijn Fibre Channel en iSCSI (met Ethernet). Hoewel de onderliggende technieken verschillen is er niet veel verschil in hoe een SAN opgezet moet worden. We zullen dan ook eerst de logische opbouw van een SAN bespreken voordat we ingaan op Fibre Channel en iSCSI.

