Tape is een van de oudste opslagsystemen voor computers die nu ook nog gebruikt wordt. Tape is een plastic lint dat voorzien is van een fijn magnetisch materiaal waardoor er door een schrijfkop enen of nullen op gezet kunnen worden. Een leeskop kan die enen en nullen weer terug lezen. Een tape wordt gelezen of geschreven door een tapestreamer. Afhankelijk van de tapestreamer kan deze geladen worden met 1 of meer tapes. Data wordt geschreven van het begin naar het eind van de tape. Om data te lezen van het begin van een bepaalde lokatie op de tape moet de tape eerst vooruit of achteruit gespoeld worden totdat deze bij de juiste lokatie is. Daarna kan de data gelezen worden.

Tape kent een aantal voordelen:
\begin{itemize}
\item Relatief lage kosten van de tapes
\item Oneindig uitbreidbaar, we kunnen steeds nieuwe tapes kopen
\item Relatief lange levensduur
\item Energie zuinig, tapes die niet meer in de drive zitten hebben geen spanning nodig om hun data te behouden
\end{itemize}

Natuurlijk zijn er ook nadelen aan tapes:
\begin{itemize}
\item Ze zijn relatief traag
\item Het kan wat werk zijn om de juiste tape terug te vinden
\item Tapes maken gebruik van magnetisme om hun data op te slaan, dus zijn ze gevoelig voor bitrot
\item Omdat ze data kunnen verliezen zullen tapes eens in de zoveel tijd geconroleert moeten worden of de data nog toegankelijk is
\end{itemize}

