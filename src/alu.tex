De Arithmetic Logical Unit is \'e\'en van de kloppende harten van een CPU. Het is de feitelijke verwerkingsunit. De ALU kent drie inputs:
\begin{itemize}
\item Operand A
\item Operand B
\item Uit te voeren instructie (I)
\end{itemize}
Daarnaast heeft een ALU twee outputs:
\begin{itemize}
\item Het resultaat van A instructie B (R)
\item De status (S) van de uitgevoerde instructie; goed of fout
\end{itemize}
Een functie kan een rekenkundige functie zijn zoals optellen of vermenigvuldigen, maar het kan ook een logische functie zijn zoals een AND of een XOR.

Een ALU kan alleen met gehele getallen (integers) werken, niet met getallen waarin een komma voorkomt (floating point).

