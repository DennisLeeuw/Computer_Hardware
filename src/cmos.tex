De CMOS (Complementary Metal Oxide Semiconductor) wordt in de computer gebruikt om de instellingen van de hardware configuratie te bewaren dus, als jij in je BIOS aanzet dat je ventilatoren altijd op 100\% moeten draaien dan onthoudt het CMOS dat. CMOS wordt gebruikt omdat het heel weinig energie verbruikt. Het kan met een simpele batterij jarenlang data bewaren zonder dat de batterij leeg raakt.

Ook houdt het CMOS bij hoe laat het is en welke datum het is. Deze data wordt niet alleen opgeslagen maar ook bijgehouden.

