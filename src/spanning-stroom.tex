Binnen de elektronica streken we van spanning en stroom zoals we met water spreken van druk en stroomsnelheid. Houden we een vat met water boven ons hoofd en zetten we de kraan open dan zal er water uit het vat stromen omdat de druk van het water in het vat het water er bij de kraan uitdrukt.

Bij elektriciteit is dat hetzelfde de spanning (druk) zorgt ervoor dat er een stroom kan lopen. Bij elektriciteit moet de stroom in een kring rond kunnen lopen om te kunnen werken. Een stroomkring moet dus altijd gesloten zijn.

De stroom loopt van de plus naar de min. Dit is een historische fout waar we nog altijd gebruik van maken. In werkelijkheid bewegen electronen van de negatief naar positief. Feitelijk maakt het niet veel uit of we een positieve lading hebben die van plus naar min loopt of een negatieve lading die van min naar plus loopt.

