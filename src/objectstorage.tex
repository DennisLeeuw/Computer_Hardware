Met het groeien van filesystemen wordt het steeds moeilijker om data aan te bieden. Op het moment dat we honderden terabytes of zelfs petabytes aan opslag hebben hebben we ook enorm veel bestanden. Mensen verdelen de data die zij opslaan niet evenredig over een bestandssysteem, sommige mappen bevatten veel bestanden en andere heel weinig. Het indexeren van al deze bestanden, als we bijvoorbeeld een listing opvragen van het bestandssysteem, duurt met de groei van het systeem steeds langer. De oplossing voor dit probleem is het niet meer gebruik maken van een bestandssysteem, maar van andere technieken. Een van die technieken is het gebruik van object storage.

Object storage wordt voornamelijk gebruikt als backend voor applicaties. Als je een bestand in een object storage systeem opslaat krijg je van het systeem een ID terug. Meestal in de vorm van een URL, bijvoorbeeld:
\texttt{https://my.nextcloud.local/index.php/s/xLiiLba2gximHBt}
Dit is een verwijzing naar het object op het systeem. Zoals je kan zien is dit niet lekker makkelijk te onthouden, vandaar dat dit vaak door applicaties gebruikt wordt, die dan een database gebruiken om een relatie te leggen tussen het object en het door de gebruiker opgeslagen bestand.
