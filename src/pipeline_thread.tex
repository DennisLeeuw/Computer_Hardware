Een core handelt instructies \'e\'en voor \'e\'en af. Als we een simpel stappen plan (thread, pipeline) nemen:
\begin{enumerate}
\item Fetch operands
\item Fetch instruction
\item Decode instruction
\item Execute instruction
\item Write output of instruction to RAM
\end{enumerate}
Dan kunnen we per clock tick een stap nemen. We kunnen ook per clock tick twee stappen doen. We kunnen namelijk bij de opgaande vlank van de clock een instructie doen en \'e\'en bij de neergaande vlank. Dit heet superpipelining. Een stap bij de opgaande en bij de neergaande vlank heet een sub-stage. Er zitten dus 2 sub-stages in 1 clock cycle. Met sub-stage maken we de processor twee keer zo snel.

Als we deze processen parallel doen in meerdere cores kunnen we spreken van een Superscalar Architecture.

Met voldoende registers kunnen we nog meer stappen tegelijk doen:
\begin{description}
\item [Clock tick 1 opgaande vlank] Haal operand A1 uit RAM en plaats in register A (adresbus, databus en controlbus bezet, register A bezet)
\item [Clock tick 1 neergaande vlank] Haal operand B1 op uit RAM en plaats in register B (adresbus, databus en controlbus bezet, register A en B bezet)
\item [Clock tick 2 opgaande vlank] Haal de instructie 1 op uit RAM en plaats in register C (adresbus, databus en controlbus bezet, register A, B en C bezet)
\item [Clock tick 2 neergaande vlank] Decodeer instructie 1 en haal operand A2 op uit RAM en plaats in register D (adresbus, databuse en controlbus bezet, register A,B,C en D bezet)
\item [clock tick 3 opgaande vlank] Voer instructie 1 uit en plaats resultaat en de status in registers E en F en we halen operand A2 op uit RAM en plaatsen het in register G (adresbus, databus en controlbus bezet, register D,E,F en G bezet)
\item [clock tick 3 neergaande vlank] Haal instructie 2 op uit RAM en plaats in register C (adresbus, databus en controlbus bezet, register C,D,E,F en G bezet)
\item [clock tick 4 opgaande vlank] Decode instructie 2 en write output instructie 1 naar RAM (adresbus, databus en controlbus bezet, register C,D en G bezet)
\item [Clock tick 4 neergaande vlank] Voer instructie 2 uit op registers D en G en stop resultaat en status in A en B, haal operand A3 op uit RAM en plaats in E (adresbus, databus en controlbus bezet, registers A,B en E bezet)
\item etc.
\end{description}
Op deze manier kunnen er meerdere instructies tegelijk uitgevoerd worden. Binnen een externe clock tick gebeurt er in de CPU meer dan je bij 1 clock tick zou verwachten. Ook kunnen opdrachten misschien wel vast uitgevoerd worden op de FPU terwijl de ALU bezig is, zo kan er nog veel meer parallel.

Met meer cores kunnen we threads of pipelines ook nog eens verdelen over de cores zodat we kunnen spreken over Hyper-threading (HT of HTT) bij Intel CPUs of over Simultaneous Multithreading (SMT) bij AMD CPUs.

Software moet instaat zijn om met HTT of SMT om te gaan. Vaak kan de techniek in de BIOS uit gezet worden. Het voordeel is dat de CPU minder idle is, maar er zijn meer registers nodig. Er zijn hacks bekend die gebruik maken van HTT om security informatie buit te maken, dus soms wordt er geadviseerd om HTT uit te zetten.

