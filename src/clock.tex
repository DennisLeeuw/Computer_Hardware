Een computer heeft net als een mens een hart nodig. Het ritme van het computerhart bepaalt de snelheid waarop de computer kan werken. Het hart in de computer is een kristal. Als je een spanning op een kristal zet gaat deze in een bepaalde frequentie trillen. Deze frequentie wordt gebruikt als heartbeat in de computer. In de computer is de heartbeat een blokgolf.

Via chips en gebruik makend van de opgaande en neergaande flank van de clock frequentie kunnen er binnen de computer verschillende snelheden gebruikt worden om bepaalde processen aan te sturen.

