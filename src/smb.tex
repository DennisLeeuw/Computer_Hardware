SMB maakt het mogelijk om bestanden en printers te delen met het netwerk en heeft tevens een interproces communicatie mogelijkheid in de vorm van een door Microsoft ontwikkelde versie van RPC genaamd MSRPC.

Het SMB-protocol stamt uit 1983 en was oorspronkelijk ontwikkeld bij IBM. Microsoft gebruikte een verder door ontwikkelde versie in zijn Windows Operating System. SMB1 die in 2003 ook bekend was onder de naam CIFS (Common Internet File System) was een erg chatty protocol, waardoor het veel bandbreedte in gebruik nam en dus niet erg geschikt was voor gebruik over Internet links. Voor het delen van lokale resources werd het echter veel gebruikt. SMB1 zou inmiddels niet meer gebruikt moeten worden omdat het vele kwetsbaarheden bevat. SMB2 verminderde de chattyness van het protocol 
