Vul een plastic buis met pingpong ballen. Druk er nog een extra bal in en neem waar dat er 
\begin{enumerate}
\item Kracht nodig is om de pingpong bal erin te drukken
\item Bij elke extra bal die er bij de ingang in gaat er een bal bij de uitgang uitkomt
\end{enumerate}

De kracht die je moet gebruiken om er een pingpongbal in te drukken is vergelijkbaar met de spanning die nodig is om elektronen te laten bewegen.

Het verlies van pingpong ballen aan het uiteinde is natuurlijk zonde. Als je de pingpongbal weer opraapt en er aan het begin weer instopt, dan scheelt dat een hoop pingpongballen. Zo werkt het in de natuur ook. Elektronen lopen rond, je hebt dus een stroomkring nodig voor elektriciteit.
