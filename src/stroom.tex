Elektriciteit is het verplaatsen van elektronen door een materiaal. Zie het als slang met water. Als je die aansluit op de kraan en de kraan open zet dan zal er aan het andere uiteinde van de slang water uitkomen. Zo werkt elektriciteit ook, alleen is het niet water dat wordt verplaatst maar elekronen.

Elektronen zijn negatief geladen deeltjes en dus lopen ze van de negatieve (-) kant naar de positieve (+) kant. Dit bewegen van negatieve deeltjes noemen we stroom\index{stroom}. Helaas wisten de wetenschappers toen ze elektrotechniek ontdekten niets van elektronen en hebben de ze bepaald dat stroom de beweging van + naar - is. Het maakt voor de werking van de elektrotechniek niet uit of je fictieve positieve deeltjes hebt die van de plus naar de min lopen of werkelijke deeltjes die van de min naar de plus lopen. Het effect is hetzelfde, er wordt lading verplaatst.

De stroom heeft als symbool de I\index{I} en wordt uitgedrukt in Amp\`ere\index{Amp\`ere}, afgekort de A\index{A}. De we spreken van een stroom van bijvoorbeeld 5 A. \[ I = 5 A \].
