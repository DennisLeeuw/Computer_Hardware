Elektriciteit is het verplaatsen van geladen deeltjes door een materiaal. In de meeste gevallen zijn het elektronen die door het materiaal bewegen.

De bouwstenen van alle materialen zijn atomen. Atomen zijn de kleinste deeltjes die nog de eigenschap van een materiaal bevatten. Het kan nog kleiner, maar dan veranderen ook de eigenschappen. Een atoom is opgebouwd uit protonen, elektronen en mogelijk ook neutronen. Zoals de naam al aangeeft zijn neutronen neutraal, dat wil zeggen dat ze geen lading hebben. Protonen en elektronen hebben wel een lading. Protonen zijn positief geladen deeltjes en elektronen negatief geladen deeltjes. Net als bij magneten de noord-pool de zuid-pool aantrekt, trekken de positieve deeltjes de negatieve deeltjes aan.

Stoffen opgebouwd uit atomen lijken voor ons soms heel massief en hard, maar in werkelijkheid bestaan ze uit vrij veel lege ruimte. Een atoom bestaat uit een atoomkern met daar omheen elektronen. Net zoals de planeten om de zon draaien. De atoomkern bestaat uit protonen en neutronen en is behoorlijk massief, maar de elektronen draaien op relatief veel afstand van de kern. Het zijn deze elektronen die kunnen bewegen en van atoom naar atoom kunnen stromen.

Leggen we nu een reeks koperatomen op een rij dan hebben we een koperdraad. Als die 1 atoom dik zou zijn dan wordt het allemaal wat heel erg klein, maar als we de dikte ook van meerdere atomen maken dan begint het al meer op stroomdraad te lijken. Knippen we deze draad in 2 gelijke stukken en verbinden we de ene kant van elke draad met een lamp en de andere met de + of de - zijde van een batterij dan kunnen de elektronen van de - kant van de batterij door de koperdraad naar de lamp bewegen en van de lamp door de andere koperdraad weer terug naar de + kant van de batterij. We kunnen dan dus elektronen doorgeven. Het doorgeven van elektronen gebeurt in een stroomkring.

Elektronen zijn negatief geladen deeltjes en die stromen van de negatieve (-) kant naar de positieve (+) kant. De wetenschappers wisten toen ze elektrotechniek ontdekten niets van elektronen en hebben toen bepaald dat stroom de beweging van positieve deeltjes is van + naar -. Het maakt voor de werking van de elektrotechniek niet uit of je fictieve positieve deeltjes hebt die van de plus naar de min lopen of werkelijke negatieve deeltjes die van de min naar de plus lopen. Het effect is hetzelfde, er wordt lading verplaatst.

De stroom heeft als symbool de I\index{I} en wordt uitgedrukt in Amp\`ere\index{Amp\`ere}, afgekort de A\index{A}. We spreken van een stroom van bijvoorbeeld 5 A. \[ I = 5 A \].

