Harddisks zijn in het verleden voornamelijk met de IDE (Integrated Drive Electronics) interface aan het moederbord gekoppeld. Later kwam er de EIDE (Extended IDE) interface. De interface staat ook bekend als Parallel Advanced Technology Attachement (PATA) interface. Het is een bus die alleen binnen een computerkast gebruikt kan worden. De IDE-kabel bestond uit een platte kabel (flat-ribbon cable) met verschillende draden naast elkaar, vandaar Parallel in PATA, zonder een afscherming. Het signaal door de kabel is dan ook erg gevoelig voor verstoringen. Op de kabel zitten 3 connecoren. 1 voor het moederbord en twee om harddisks aan te koppelen. Zo kunnen er een master en een slave drive aangesloten worden op de kabel.

Een moederbord had in het verleden meestal twee host adapters, IDE1 en IDE2 ofwel de primary en de secondary IDE. Dit maakt het totaal aan te sluiten disks 4. In de eerste instantie moest er een jumper op de disk gezet worden om aan te geven of deze master of slave was. De meeste kabels zijn echter \textquote{Cable Select} en bepalen aan de hand van de positie van de disk of deze master of slave is. Degene die het dichtst bij het moederbord zit is de master.

De IDE-bus is 16-bits breed op een initieel 40-pins connector. Met de wens voor hogere snelheden kwam met UDMA4 de eis dat de kabel afgeschermd zou zijn waardoor de 40-aderige kabel vervangen werd door een 80-aderige afgeschermde kabel. Draad en dus pin 1 op de kabel is rood, of rood gestreept. De connectoren zijn voorzien van een nokje waardoor deze maar op \'e\'en manier in de drive of het moederbord passen. Een IDE-kabel kan maximaal 46cm lang zijn.

