De computer heet een computer omdat hij kan rekenen (to compute), maar de echte rekenaar van de computer is de CPU de rest van de computer is er alleen om ons, gebruikers, te ondersteunen. De functie van de CPU is de verwerking van instructies en data. Het doet dit door een enorme hoeveelheid transistoren die ge\"intergreerd zijn op een stuk gezuiverd silicium (silicon). Het geheel wordt geproduceerd op een wafer, wat een grote ronde plaat is met vele toekomstige CPU's erop. De wafer wordt in stujes gebroken tot de zogenaamde \textquote{dies}. Elke \textquote{die} wordt verplakt in een behuizen en verkocht als de CPU of processor.

Moderne processoren zijn complexe chips waarop veel meer  ge\"integreerd is dan alleen de rekeneenheid (ALU). Het is een samenraapsel van allerlei functies die samen op een stuk silicium (Engels: die) zijn samengebracht. De reden dat het een samenraapsel is is omdat er in de loop van de tijd steeds meer functies van het moederbord zijn overgebracht naar de processor omdat dat het geheel sneller maakt.

