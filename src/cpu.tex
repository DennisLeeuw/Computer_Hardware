De processor (ook wel CPU of Central Processing Unit genoemd) vormt het hart van de computer. Met een centrale plek op het moederbord en een heleboel aansluitingen alle kanten op, heeft het supersnel toegang tot alles waar maar toegang tot te krijgen is in een computer. 

Het is het onderdeel van de computer dat ‘denkt’. De vraag is of je een computer wel een computer kunt noemen als er geen processor in zit. Hij kan immers geen berekeningen maken, niet  ‘denken’, geen programma uitvoeren (zelfs geen piepklein simpel programmaatje), geen besturingssysteem zoals Windows draaien... eigenlijk gewoon helemaal niks. Als je in een computer kijkt, kun je de processor nooit zien zitten. Wat je wél ziet is de grote actieve koeler die bovenop de processor zit om te voorkomen dat hij oververhit raakt tijdens al het rekenen. 
