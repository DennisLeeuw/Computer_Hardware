Het is mogelijk om een CPU op een hogere snelheid te laten werken dan waarvoor de fabrikant hem verkocht heeft. CPU's worden gemaakt op gezuiverd silicium, maar dat is nooit 100\% zuiver. Kleine verontreinigingen zorgen voor kleine verschillen tussen de CPU's van \'e\'en wafer. De fabrikant test de verschillende dies en beslist dan voor welke clock snelheid de die verkocht kan worden. Over het algemeen is dat de snelheid waarbij er voor 100\% zekerheid gesteld kan worden dat er geen fouten optreden. Daar zit natuurlijk een veiligheidsmarge op. De fabrikant wil geen claim aan zijn broek krijgen van een bank dat er een miljarden overschrijving fout is gegaan.

Voor thuis kan een iets hogere snelheid met het risico op kleine foutjes echter wel acceptabel zijn. Dit op een hogere snelheid laten werken, dan waarvoor de CPU verkocht werd, heet overclocking. Er zijn verschillende manieren waarop overclocking bereikt kan worden:
\begin{description}
\item[Core speed overclocking] Zorgen dat de snelheid waarmee de CPU rekent (ALU, FPU) omhoog gaat
\item[FSB overclocking] Zorgen dat de data sneller van en naar het RAM gaat
\item[beide] Gebruik van beide technieken
\end{description}

Als de CPU meer berekeningen per seconde moet doen, moet hij meer arbeid verzetten en dus meer vermogen verbruiken wat ook betekent dat hij meer warmte produceerd. We zullen dus beter moeten koelen en rekening houden met extra elektriciteit verbruik.

