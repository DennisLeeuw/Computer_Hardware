De AND\index{AND} geeft aan wanneer beide ingangen 1 zijn, alleen dan is de uitgang 1.

\rowcolors{2}{gray!10}{gray!20}
\begin{tabular}{ |c|c|c| }
\hline
\rowcolor{gray!60}
	Input 1 & Input 2 & Output \\
	\hline
	0 & 0 & 0 \\
	\hline
	0 & 1 & 0 \\
	\hline
	1 & 0 & 0 \\
	\hline
	1 & 1 & 1 \\
	\hline
\end{tabular}

Het symbool voor de AND is weergegeven in figuur \ref{symbool:and}

\begin{figure}[h]
\includegraphics{and_symbool}
\centering
\caption{Symbool van een AND}
\label{symbool:and}
\end{figure}

De AND wordt gebouwd door gebruik te maken van twee transistoren waarvan de beide basis de ingang vormen en een weerstand (figuur \ref{circuit:and}.

\begin{figure}[h]
\includegraphics{and_circuit}
\centering
\caption{AND circuit}
\label{circuit:and}
\end{figure}

