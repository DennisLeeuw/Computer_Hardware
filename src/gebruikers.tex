Als we stroom en spanning vergelijken met een rivier dan is het hoogte verschil bepalend voor de snelheid waarmee de rivier zou kunnen stromen. De bodem van de rivier remt het water af, we noemen dat de weerstand. Er zijn nog andere manieren om de rivier af te remmen en dat is door er een schoepenrad in te zetten. Een schoepenrad in de rivier verbruikt het water niet, maar zorgt er wel voor dat de rivier minder hard stroomt. Het schoepenrad gebruikt de rivier om arbeid te verrichten, bijvoorbeeld om graan te malen.

In de electronica kennen we ook gebruikers, of verbruikers. Gebruikers zorgen ervoor dat de stroom minder snel stroomt. Er wordt dus weerstand geboden aan de stroom. We zeggen dan ook dat bijvoorbeeld een lamp die brandt er voor zorgen dat de weerstand verhoogd wordt. Een gebruiker verbruikt geen stroom, maar zorgt er wel voor dat er in het totale circuit minder stroom loopt.

