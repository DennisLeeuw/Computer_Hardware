Block devices\index{Block devices} zijn opslagsystemen die data opslaan in blokken. Een harddisk slaat data bijvoorbeeld op in blokken van 512-bytes en is dan ook een block device. Data die wordt weggeschreven naar een block device wordt dus eerst opgehakt in kleine vaste stukken en dan per block weggeschreven. Het tegenover gestelde systeem is een stream, daarin wordt data als \'e\'en lange rij 1-en en 0-en weggeschreven. Bij een stream staat de data dus keurig netjes achter elkaar. Bij block devices kan de data verspreid staan over het opslag medium.
