Bij het gebruik van energie gaat het uiteindelijk om het verrichten van arbeid. Als we een electrische auto laten rijden wordt er arbeid verricht, zo ook bij het branden van een lamp en bij het zetten van koffie met een koffiezetapparaat. Electrische energie wordt omgezet in een energie vorm die we op dat moment nodig hebben. Bij een lamp wordt electrische energie omgezet in licht, bij auto in beweging en bij een koffiezetapparaat in warmte om het water aan de kook te brengen.

Om electriciteit deze arbeid te laten verrichten hebben we een stroomkring nodig. De energie gaat van de energie opwekker, naar de verbruiker en weer terug naar de opwekker.

Het is te vergelijken met een schoepenrad in een rivier. De zon verwarmt de zee en verdampt zo water. Water drijft in wolken naar de kust. Boven het land valt het water naar beneden in de vorm van regen, sneeuw of hagel. Uiteindelijk komt het water in een rivier terecht en door de zwaartekracht wordt het stroomafwaarts vervoert naar de zee. De kringloop kan opnieuw beginnen. Als we gebruik willen maken van de energie van de rivier dan kunnen we een schoepenrad plaatsen en via een as bijvoorbeeld een molensteen aandrijfen om graan te malen. Er blijft dezelfde hoeveelheid water door de rivier stromen, en toch hebben we bewegingsenergie van de rivier gebruikt om graan te malen.

In de electrotechniek gebeurt hetzelfde. Het water uit de rivier is te vergelijken met de stroom die loopt door een stroomkring. Een batterij is een stroombron, dus die levert stroom, via een koperdraad gaat dit naar een lamp, die geeft licht, en van de lamp gaat de stroom via een koperdraad terug naar de batterij. Bij de rivier is het de zwaartekracht die ervoor zorgt dat het water stroomt en bij de batterij is het de spanning die ervoor zorgt dat de stroom stroomt.

De hoeveelheid arbeid die verricht wordt door de lamp die electriciteit omzet in licht is het vermogen dat gebruikt wordt en vermogen wordt uitgedrukt in Watt. In de winkel staat dan ook altijd op een lamp hoeveel Watt deze is.

We kunnen het vermogen van de rivier op twee manieren be\"invloeden. We kunnen er meer water doorsturen, of we kunnen een stijlere rivier nemen zodat het water sneller stroomt. Ditzelfde geldt voor electriciteit, we kunnen meer stroom laten vloeien door de spanning te verhogen. Hieruit blijkt dat spanning, stroom en vermogen van elkaar afhankelijk zijn. Die afhankelijkheid kunnen we weergeven in een formule:

\[ P = U*I \]

P = Vermogen
U = Spanning
I = Stroom

Als ik de spanning verdubbel, bij een gelijk blijvende stroom, verdubbelt dus het vermogen. Zo geldt ook dat als ik het vermogen constant hou en de spanning verdubbel, de stroom dus moet halveren.

Een andere manier om meer stroom te laten lopen is door de weerstand te verlagen. De weerstand is te vergelijken met de obstakels in een rivier. Liggen er heel veel stenen in de rivier waar het water op botst dan zal de stroomsnelheid afnemen. Nemen we de stenen weg dan zal de rivier sneller stromen. In de electronica geldt hetzelfde. In onze eenvoudig stroomkring met een batterij, twee stukken koperdraad en een lamp kunnen we niet zo veel doen om de weerstand te verlagen, we kunnen hem echter wel eenvoudig verhogen. Door een weerstand op te nemen in het circuit kunnen we een extra \textquote{steen} in de stroomkring aanbrengen zodat er minder stroom loopt.

Dus door een weerstand toe te voegen loopt er bij gelijk blijvende spanning minder stroom. Hieruit mogen dus vaststellen dat er een relatie is tussen spanning, stroom en weerstand. En die is er ook, die relatie heet de wet van Ohm:

\[ U = I*R \]

U = Spanning
I = Stroom
R = Weerstand

Door de weerstand in het circuit loopt er minder stroom, de lamp kan dus minder arbeid verrichten en zal minder fel gaan branden. Het vermogen (verbruik van de lamp) neemt dus af. Als we willen weten wat het vermogen van de lamp is nadat we de weerstand hebben toegevoegd dan moeten we de twee voorgaande formules combineren:

\[ P = U^2/R \]

P = Vermogen
U = Spanning
R = Weerstand

Deze combinatie wordt gemaakt door substitutie. Substitutie is een moeilijk woord voor vervanging. In de formule P=U*I hebben we de I vervangen door de I uit U=I*R. Om te weten wat de I is uit U=I*R moeten we delen door hetzelfde "getal" aan beide kanten van het = teken. Dus als we I*R delen door R dan houden we I over. Nu moeten we ook delen door R aan de U kant, we houden dus U/R over:

U/R = I

Nu we weten wat I is kunnen we dat in de formule voor het vermogen stoppen:

\[ P = U*(U/R) = U^2/R \]

En we zien nu dat bij een gelijkblijvende spanning, het vermogen van de lamp direct afhankelijk is van de weerstand. Als we de weerstand 2x zo groot maken zal het vermogen 2x zo klein worden. Er is dus een omgekeerde afhankelijkheid tussen het vermogen en de weerstand.

Omdat we in de elektronica elektronen door materiaal duwen, wordt er arbeid\index{Arbeid} verricht en deze arbeid wordt voor een bepaalde tijd verricht. De P\index{P} heeft dan het vermogen en de eenheid die daar bij hoort is de Watt\index{Watt} (W\index{W}). Als we dit vermogen een uur lang gebruiken dan spreken we van een Wh\index{Wh} (Watt-uur). Bij 1000 Wh korten we dit af tot een kWh\index{kWh} ofwel een kilo-Watt-uur.

