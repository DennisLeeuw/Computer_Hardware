Omdat we in de elektronica elektronen door materiaal duwen, wordt er arbeid\index{Arbeid} verricht en deze arbeid wordt voor een bepaalde tijd verricht. We kunnen de verbruikte arbeid uitdrukken in een formule en daar komt dan het vermogen\index{Vermogen} uit: \[ P = U*I \]
De P\index{P} heeft dan het vermogen en de eenheid die daar bij hoort is de Watt\index{Watt} (W\index{W}).

Als we dit vermogen een uur lang gebruiken dan spreken we van een Wh\index{Wh} (Watt-uur). Bij 1000 Wh korten we dit af tot een kWh\index{kWh} ofwel een kilo-Watt-uur.

