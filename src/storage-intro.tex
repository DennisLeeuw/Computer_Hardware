RAM-geheugen is volatile, dat betekent dat de inhoud van het geheugen verloren gaat als de voeding wordt uit gezet. Documenten, programma's en het besturingssysteem willen we niet steeds opnieuw maken. We willen dat deze data opgeslagen wordt op een systeem dat de data ook bewaart als er geen spanning op de computer staat. De oplossing die daarvoor bedacht is heet opslag-geheugen of kortweg opslag (Engels: Storage).

Onder Storage verstaan we het aanbieden van opslagcapaciteit op opslagsystemen. Dit kan op harddisks in het systeem zijn of op een fileserver die gekoppeld is aan het netwerk. Storage is dus een heel breed onderwerp. In dit document behandelen we alleen de Storage die in een computer zit of die direct gekoppeld is via bijvoorbeeld de USB-poort. Het document over NAS, Network Attached Storage, en het document over SAN, Storage Attached Network, behandelen de andere vormen van opslag. Ook voor RAID hebben we ervoor gekozen om dit in een eigen document te behandelen.

