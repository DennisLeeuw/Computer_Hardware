De elektrische stroom loopt door de stroom draad, maar eigenlijk zijn het elektronen die door de draad bewegen. Het bewegen door de draad kost moeite. In de draad zit een bepaalde weerstand\index{Weerstand} die door de elektronen overwonnen moet worden. Door een potentiaal verschil (spanning) aan te leggen kunnen we de elektronen door de draad drukken. Er is dus een relatie tussen de spanning, de stroom en de weerstand van de kabel.

In 1826 toonde George Ohm\index{Ohm} deze relatie aan en vatte die in een formule die naar hem genoemd is: De wet van Ohm\index{Wet van Ohm}. De formule die daarbij hoort is \[ R = \frac{U}{I} \]

R is hierbij de weerstand die wordt uitgedrukt in aantallen ohm ($\Omega$). Als er bij een spanning van 1,5 V een stroom door de draad loopt van 2 A, dan geldt voor de weerstand van die draad: \[ R = \frac{1,5}{2} = 0,75 \Omega \]

De formule wordt vaker geschreven als: \[ U = I*R \]

